\documentclass[10pt]{article}
\usepackage[margin=0.25in]{geometry}
\usepackage{enumitem}
\usepackage{titlesec}
\usepackage{parskip}
\usepackage{hyperref}
\usepackage{multicol}
\usepackage{fancyhdr}
\usepackage[dvipsnames]{xcolor}

% Define custom color
\definecolor{mygreen}{HTML}{004225}
\newcommand{\sectioncolor}{\color{mygreen}}

\pagestyle{empty}

% Line between sections (no spacing)
\newcommand{\sectionline}{\noindent\rule{\linewidth}{0.4pt}}

% No extra paragraph spacing
\setlength{\parskip}{0pt}
\setlength{\parindent}{0pt}

% Tight bullet formatting
\setlist[itemize]{left=1.25em, itemsep=0pt, topsep=0pt, parsep=0pt}

% Tight section formatting with color
\titleformat{\section}
{\sectioncolor\bfseries\uppercase}
{}{0pt}{}
\titlespacing{\section}{0pt}{2pt}{2pt}

\begin{document}
	
	\begin{center}
		{\color{mygreen}
			{\Huge \textbf{Scott Nguyen}} \\
			\small
			\href{mailto:scott.nguyen8901@gmail.com}{scott.nguyen8901@gmail.com} \textbar
			(515) 943-6283 %\textbar
			%\href{https://www.linkedin.com/in/scottnguyen8901/}{linkedin.com/in/scottnguyen8901} \textbar
			\textbf{Security Clearance: Interim Secret}
		}
	\end{center}

	\sectionline
	
	\section*{Education}
	
	\textbf{Master of Science in Electrical Engineering \textbar Space Systems Engineering} \hfill \textbf{Summer 2026} \\
	University of New Mexico \hfill GPA: 3.94/4.00
	
	\textbf{Master of Science in Aerospace Engineering} \hfill \textbf{Fall 2024} \\
	University of Illinois Urbana-Champaign
	
	\textbf{Bachelor of Science in Aerospace Engineering} \hfill \textbf{Spring 2022} \\
	Iowa State University
	
	\sectionline
	
	\section*{Skills}
	\textbf{Programming Languages:} MATLAB, Python, C/C++, Ruby \\
	\textbf{Frameworks / Libraries / Tools:} NumPy, SciPy, Matplotlib, Astropy, Poliastro, bpy, git \\
	\textbf{Applications:} Simulink, Blender
	
	\sectionline
	
	\section*{Work Experience}

	\textbf{Student Co-Op, Electronics for Contested Space Group} \hfill \textbf{September 2025 – Present} \\
	\emph{MIT Lincoln Laboratory}
	\begin{itemize}
		\item Implemented an \textbf{\emph{Unscented Kalman Filter (UKF)}} for precise optical sensor measurement fusion and state estimation
		\item Built a probabilistic detection tool to compute observation likelihoods based on resident space object properties and optical sensor performance
	\end{itemize}

	\textbf{Guidance, Navigation \& Controls Engineer Intern} \hfill \textbf{May 2025 – August 2025} \\
	\emph{Blue Canyon Technologies}
	\begin{itemize}
		\item Performed post-environment (TVAC/vibe) functional testing of \textbf{\emph{IMU}}, \textbf{\emph{Nano Star Tracker}}, \textbf{\emph{Reaction Wheels}}, \textbf{\emph{Torque Rods}}, and \textbf{\emph{Sun Sensors}}; verified polarity, health, and performance against flight acceptance criteria.
		\item Led root-cause investigation of \textbf{\emph{SADA}} command–response latency; characterized timing delay across modes, identified the source, and informed updates to command/telemetry interfaces and test limits.
		\item Automated \textbf{\emph{SADA}} validation in \textbf{\emph{COSMOS}} with \textbf{\emph{Ruby}} scripts; mapped telemetry channels, implemented pass/fail logic, and generated reproducible test reports.
	\end{itemize}

	\textbf{Guidance, Navigation \& Controls Engineer Intern} \hfill \textbf{January 2025 – April 2025} \\
	\emph{Blue Origin}
	\begin{itemize}
		\item Designed a hybrid \textbf{\emph{Active Disturbance Rejection Control (ADRC)}} + \textbf{\emph{Sliding Mode Control (SMC)}} controller; used an extended state observer and a sliding manifold for robust tracking under uncertainty and actuator limits.
		\item Injected time-domain disturbances and conducted a trade study comparing the hybrid controller to \textbf{\emph{PID}} and \textbf{\emph{LQR}} on LTI linearizations; benchmarked rise time, settling time, and overshoot.
		\item Integrated flight software into \textbf{\emph{Simulink}} via \textbf{\emph{C}} \textbf{\emph{S-functions}} to run full closed-loop simulations.
	\end{itemize}

	\textbf{Guidance, Navigation \& Controls Engineer Intern} \hfill \textbf{May 2024 – August 2024} \\
	\emph{Varda Space Industries}
	\begin{itemize}
	    \item Conducted trade studies to optimize gravity models for mission requirements and select the best filter (\textbf{\emph{Extended Kalman Filter (EKF)}} vs. \textbf{\emph{Unscented Kalman Filter (UKF)}})
	    \item Created Monte Carlo simulations to perform flight safety analysis and develop reentry criteria for capsule reentry
	    \item Implemented an \textbf{\emph{EKF}} for state estimation, optimizing ground station data timing to minimize residuals and enable precise delta-v planning
	    \item Implemented unit testing for the code base and introduced CI/CD pipelines using Bamboo
	\end{itemize}

	\textbf{Guidance, Navigation \& Controls Engineer Intern} \hfill \textbf{May 2023 – August 2023} \\
	\emph{Space Dynamics Laboratory}
	\begin{itemize}
	    \item Implemented a UKF with range iteration and least squares orbit determination methods using optical navigation
	    \item Performed Monte Carlo analysis on relative orbits to identify challenging scenarios and refine the algorithm
	    \item Programmed and developed unit tests for \textbf{\emph{Lambert Solver}} to be utilized with \textbf{\emph{Initial Orbit Determination (IOD)}}
	\end{itemize}
	
	\sectionline
	
	\section*{Research Projects}
	
	\textbf{Delta-V Minimization from Geostationary Orbit to Mars}
	\begin{itemize}
		\item Generated pork-chop plots using Lambert solutions and cross-validated optimizer results with the plot’s global and local minimum regions to ensure consistency
		\item Utilized Blender’s Python API to visualize the optimized trajectory and animate planetary motion
	\end{itemize}
	
\end{document}
