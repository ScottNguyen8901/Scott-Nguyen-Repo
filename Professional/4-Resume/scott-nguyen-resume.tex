\documentclass[10pt]{article}
\usepackage[margin=0.5in]{geometry}
\usepackage{enumitem}
\usepackage{titlesec}
\usepackage{parskip}
\usepackage{hyperref}
\usepackage{multicol}
\usepackage{fancyhdr}
\usepackage[dvipsnames]{xcolor}

% Define custom color
\definecolor{mygreen}{HTML}{004225}
\newcommand{\sectioncolor}{\color{mygreen}}

\pagestyle{empty}

% Line between sections (no spacing)
\newcommand{\sectionline}{\noindent\rule{\linewidth}{0.4pt}}

% No extra paragraph spacing
\setlength{\parskip}{0pt}
\setlength{\parindent}{0pt}

% Tight bullet formatting
\setlist[itemize]{left=1.25em, itemsep=0pt, topsep=0pt, parsep=0pt}

% Tight section formatting with color
\titleformat{\section}
{\sectioncolor\bfseries\uppercase}
{}{0pt}{}
\titlespacing{\section}{0pt}{2pt}{2pt}

\begin{document}
	
	\begin{center}
		{\color{mygreen}
			{\Huge \textbf{Scott Nguyen}} \\
			\href{mailto:scott.nguyen8901@gmail.com}{scott.nguyen8901@gmail.com} \quad \textbar \quad 
			(515) 943-6283 \quad \textbar \quad 
			\href{https://www.linkedin.com/in/scottnguyen8901/}{linkedin.com/in/scottnguyen8901}
		}
	\end{center}
	
	\sectionline
	
	\section*{Education}
	
	\textbf{Master of Science in Electrical Engineering \textbar Space Systems Engineering} \hfill \textbf{Summer 2026} \\
	University of New Mexico \hfill GPA: 3.94/4.00
	
	\textbf{Master of Science in Aerospace Engineering} \hfill \textbf{Fall 2024} \\
	University of Illinois Urbana-Champaign
	
	\textbf{Bachelor of Science in Aerospace Engineering} \hfill \textbf{Spring 2022} \\
	Iowa State University
	
	\sectionline
	
	\section*{Skills}
	\textbf{Programming Languages:} MATLAB, Python, C/C++, Ruby \\
	\textbf{Frameworks / Libraries / Tools:} NumPy, SciPy, Matplotlib, Astropy, Poliastro, bpy, git \\
	\textbf{Applications:} Simulink, Blender
	
	\sectionline
	
	\section*{Work Experience}
	
	\textbf{Guidance, Navigation \& Controls Engineer Intern} \hfill \textbf{May 2025 – August 2025} \\
	\emph{Blue Canyon Technologies}
	\begin{itemize}
	    \item Verified functionality and polarity of the IMU, Nano Star Tracker, Reaction Wheels, Torque Rods, and Sun Sensors through hardware testing and data validation
	    \item Conducted regression analysis on two-axis \textbf{\emph{Solar Array Drive Assembly (SADA)}} momentum management and tested command interfaces to ensure 			   precise control and system reliability
	    \item Developed automated test scripts in \textbf{\emph{Ruby}} and mapped \textbf{\emph{SADA}} telemetry channels to \textbf{\emph{COSMOS}}, streamlining system 		    validation and real-time monitoring
	\end{itemize}

	\textbf{Guidance, Navigation \& Controls Engineer Intern} \hfill \textbf{January 2025 – April 2025} \\
	\emph{Blue Origin}
	\begin{itemize}
	    \item Integrated \textbf{\emph{Active Disturbance Rejection Control (ADRC)}} and \textbf{\emph{Sliding Mode Control (SMC)}} to develop a robust algorithm for stabilizing the nonlinear MIMO dynamics of the BE-7 engine
	    \item Evaluated control performance by injecting various disturbances, demonstrating effective rejection and improved accuracy in setpoint tracking
	   \item Integrated flight software into \textbf{\emph{Simulink}} using \textbf{\emph{S-functions}} programmed in \textbf{\emph{C}} to enable testing and verification
	    \item Compiled findings into a technical report and presented control strategies, simulations, and integration insights
	\end{itemize}

	\textbf{Guidance, Navigation \& Controls Engineer Intern} \hfill \textbf{May 2024 – August 2024} \\
	\emph{Varda Space Industries}
	\begin{itemize}
	    \item Conducted trade studies to optimize gravity models for mission requirements and select the best filter (\textbf{\emph{Extended Kalman Filter (EKF)}} vs. \textbf{\emph{Unscented Kalman Filter (UKF)}})
	    \item Created Monte Carlo simulations to perform flight safety analysis and develop reentry criteria for capsule reentry
	    \item Implemented an \textbf{\emph{EKF}} for state estimation, optimizing ground station data timing to minimize residuals and enable precise delta-v planning
	    \item Validated GPS error using hardware-in-the-loop testing with Spirent simulation
	    \item Implemented unit testing for the code base and introduced CI/CD pipelines using Bamboo
	\end{itemize}

	\textbf{Guidance, Navigation \& Controls Engineer Intern} \hfill \textbf{May 2023 – August 2023} \\
	\emph{Space Dynamics Laboratory}
	\begin{itemize}
	    \item Implemented a UKF with range iteration and least squares orbit determination methods using optical navigation
	    \item Simulated high-fidelity dynamic models with J2 perturbations, third-body dynamics, and solar radiation pressure
	    \item Performed Monte Carlo analysis on relative orbits to identify challenging scenarios and refine the algorithm
	    \item Programmed and developed unit tests for \textbf{\emph{Lambert Solver}} to be utilized with \textbf{\emph{Initial Orbit Determination (IOD)}}
	\end{itemize}
	
	\sectionline
	
	\section*{Research Projects}
	
	\textbf{Delta-V Minimization from Geostationary Orbit to Mars}
	\begin{itemize}
		\item Applied trajectory optimization techniques to minimize delta-v for an Earth-to-Mars transfer orbit, enhancing fuel efficiency
		\item Generated pork-chop plots using Lambert solutions and cross-validated optimizer results with the plot’s global and local minimum regions to ensure consistency
		\item Utilized Blender’s Python API to visualize the optimized trajectory and animate planetary motion
	\end{itemize}
	
\end{document}
