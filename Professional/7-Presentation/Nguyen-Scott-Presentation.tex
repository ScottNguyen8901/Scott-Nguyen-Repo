%========================
% Presentation Template
%========================
\documentclass[10pt,aspectratio=169]{beamer}

%--- Theme & look ---
\usetheme[numbering=fraction,progressbar=frametitle]{metropolis} % sleek beamer theme
\metroset{block=fill} % filled blocks
\setbeamertemplate{navigation symbols}{} % hide nav icons

%--- Fonts (Times-like) ---
\usepackage{newtxtext,newtxmath} % Times/Times Math
\usefonttheme{professionalfonts}

%--- Language & encoding ---
\usepackage[utf8]{inputenc}
\usepackage[T1]{fontenc}
\usepackage[american]{babel}
\usepackage{csquotes}

%--- Graphics / figures ---
\usepackage{graphicx}
\usepackage{float}
\usepackage{caption}
\captionsetup[figure]{labelfont=bf}
\usepackage{booktabs}
\usepackage{tikz}

%--- Math & helpers ---
\usepackage{amsmath,amssymb,mathtools}
\usepackage{siunitx}
\sisetup{detect-all=true}

%--- Code listings (Python style) ---
\usepackage{listings}
\usepackage{xcolor}
\definecolor{codegray}{RGB}{240,240,240}
\lstdefinestyle{py}{
	language=Python,
	basicstyle=\ttfamily\footnotesize,
	backgroundcolor=\color{codegray},
	frame=single,
	rulecolor=\color{black},
	showstringspaces=false,
	tabsize=2,
	breaklines=true,
	keywordstyle=\bfseries,
	commentstyle=\itshape,
	numbers=left,
	numberstyle=\tiny,
	xleftmargin=1.2em
}

%--- Hyperlinks ---
\usepackage{hyperref}

%--- Title info ---
\title[Short Title]{Full Presentation Title}
\subtitle{Concise Subtitle (Optional)}
\author[Your Name]{Your Name}
\institute{Your Organization / Lab}
\date{\today}

%--- Custom commands (edit as needed) ---
\newcommand{\R}{\mathbb{R}}
\newcommand{\vect}[1]{\mathbf{#1}}

%--- Theorem-like environments ---
\usepackage{amsthm}
\theoremstyle{definition}
\newtheorem{definition}{Definition}
\theoremstyle{plain}
\newtheorem{theorem}{Theorem}
\theoremstyle{remark}
\newtheorem{remark}{Remark}

%--- Bibliography (simple numeric) ---
\usepackage[backend=biber,style=numeric,sorting=nyt,maxbibnames=6]{biblatex}
\addbibresource{refs.bib}

%========================
\begin{document}
	%========================
	
	%--- Title frame ---
	\begin{frame}
		\titlepage
	\end{frame}
	
	%--- Agenda / Outline ---
	\begin{frame}{Agenda}
		\tableofcontents
	\end{frame}
	
	%================================
	\section{Introduction}
	%================================
	
	\begin{frame}{Motivation}
		\begin{itemize}
			\item Problem context in one or two lines.
			\item Why it matters (impact, constraints).
			\item Your key contribution in this talk.
		\end{itemize}
	\end{frame}
	
	\begin{frame}{Key Takeaways}
		\begin{enumerate}
			\item Takeaway \#1 — one sentence.
			\item Takeaway \#2 — one sentence.
			\item Takeaway \#3 — one sentence.
		\end{enumerate}
	\end{frame}
	
	%================================
	\section{Background}
	%================================
	
	\begin{frame}{Core Definitions}
		\begin{definition}
			Let $x \in \R^n$. We call $x$ \emph{stable} if $\|x_k\|\to 0$ as $k\to\infty$.
		\end{definition}
		\begin{remark}
			State your conventions/assumptions early.
		\end{remark}
	\end{frame}
	
	\begin{frame}{A Useful Theorem}
		\begin{theorem}[Short Name]
			If $A$ is Hurwitz, then the system $\dot{x}=Ax$ is exponentially stable.
		\end{theorem}
		\begin{proof}[Sketch]
			Choose $V(x)=x^\top Px$ with $P\succ0$ solving $A^\top P+PA=-Q$. Then $\dot{V}=-x^\top Q x\le -\lambda_{\min}(Q)\|x\|^2$.
		\end{proof}
	\end{frame}
	
	%================================
	\section{Method}
	%================================
	
	\begin{frame}{System Model}
		\begin{block}{Discrete-Time Dynamics}
			\[
			x_{k+1} = f(x_k,u_k) + w_k,\quad y_k = h(x_k) + v_k
			\]
			with $w_k\sim\mathcal{N}(0,Q)$ and $v_k\sim\mathcal{N}(0,R)$.
			\]
		\end{block}
		\begin{itemize}
			\item Specify states, inputs, and noise models.
			\item Note approximations and sampling time.
		\end{itemize}
	\end{frame}
	
	\begin{frame}[fragile]{Algorithm (Python)}
		\begin{lstlisting}[style=py,caption={Pseudocode/Reference Implementation}]
			import numpy as np
			
			def step(x, u, dt):
			# TODO: fill in dynamics
			return x_next
			
			def run_filter(y_seq, u_seq, Q, R):
			# TODO: initialize, predict, update
			return est_states
		\end{lstlisting}
	\end{frame}
	
	\begin{frame}{Two-Column Layout}
		\begin{columns}[T,onlytextwidth]
			\column{0.52\textwidth}
			\textbf{Idea}
			\begin{itemize}
				\item Point A
				\item Point B
				\item Point C
			\end{itemize}
			
			\column{0.45\textwidth}
			\textbf{Graphic}
			\begin{figure}
				\includegraphics[width=\linewidth]{figs/diagram.pdf}
				\caption{System diagram (replace with yours).}
			\end{figure}
		\end{columns}
	\end{frame}
	
	%================================
	\section{Results}
	%================================
	
	\begin{frame}{Experimental Setup}
		\begin{itemize}
			\item Hardware/Simulation details.
			\item Datasets and metrics.
			\item Train/val/test splits; seeds.
		\end{itemize}
	\end{frame}
	
	\begin{frame}{Quantitative Results}
		\begin{table}
			\centering
			\begin{tabular}{lccc}
				\toprule
				Method & Metric 1 & Metric 2 & Time \\
				\midrule
				Baseline & 0.72 & 0.55 & 12 ms \\
				Ours     & \textbf{0.81} & \textbf{0.61} & 10 ms \\
				\bottomrule
			\end{tabular}
			\caption{Replace with your actual numbers.}
		\end{table}
	\end{frame}
	
	\begin{frame}{Figures}
		\begin{figure}
			\includegraphics[width=0.8\linewidth]{figs/plot.png}
			\caption{Main plot (insert your figure).}
		\end{figure}
	\end{frame}
	
	%================================
	\section{Conclusion}
	%================================
	
	\begin{frame}{Conclusion}
		\begin{itemize}
			\item One-line summary of contribution.
			\item Biggest limitation / assumption.
			\item Clear next step / future work.
		\end{itemize}
	\end{frame}
	
	\begin{frame}{Acknowledgments}
		\small
		Thanks to collaborators and sponsors. \par
		Contact: \href{mailto:you@school.edu}{you@school.edu} \quad
		\href{https://your-site.example}{your-site.example}
	\end{frame}
	
	%--- References (optional slide) ---
	\begin{frame}[allowframebreaks]{References}
		\printbibliography
	\end{frame}
	
	%--- Backup / Appendix ---
	\appendix
	\section{Appendix}
	
	\begin{frame}{Extra Details}
		Use appendix frames for questions: proofs, ablations, backups.
	\end{frame}
	
	%========================
\end{document}
%========================
