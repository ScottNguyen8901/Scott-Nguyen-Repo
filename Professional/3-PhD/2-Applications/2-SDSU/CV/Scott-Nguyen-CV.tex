\documentclass[10pt]{article}
\usepackage[margin=0.5in]{geometry}
\usepackage{enumitem}
\usepackage{titlesec}
\usepackage{parskip}
\usepackage{hyperref}
\usepackage{multicol}
\usepackage{fancyhdr}
\usepackage[dvipsnames]{xcolor}

% Define custom color
\definecolor{mygreen}{HTML}{004225}
\newcommand{\sectioncolor}{\color{mygreen}}

\pagestyle{empty}

% Line between sections (no spacing)
\newcommand{\sectionline}{\noindent\rule{\linewidth}{0.4pt}}

% No extra paragraph spacing
\setlength{\parskip}{0pt}
\setlength{\parindent}{0pt}

% Tight bullet formatting
\setlist[itemize]{left=1.25em, itemsep=0pt, topsep=0pt, parsep=0pt}

% Tight section formatting with color
\titleformat{\section}
{\sectioncolor\bfseries\uppercase}
{}{0pt}{}
\titlespacing{\section}{0pt}{2pt}{2pt}

\begin{document}
	
	\begin{center}
		{\color{mygreen}
			{\Huge \textbf{Scott Nguyen}} \\
			\small
			\href{mailto:scott.nguyen8901@gmail.com}{scott.nguyen8901@gmail.com} \textbar
			(515) 943-6283 \textbar
			\textbf{Security Clearance: Interim Secret}
		}
	\end{center}
	
	\sectionline
	
	\section*{Education}
	
	\textbf{Master of Science in Electrical Engineering \textbar\ Space Systems Engineering} \hfill \textbf{Expected Summer 2026} \\
	University of New Mexico \hfill GPA: 3.94/4.00
	
	\textbf{Master of Science in Aerospace Engineering} \hfill \textbf{Fall 2024} \\
	University of Illinois Urbana–Champaign
	
	\textbf{Bachelor of Science in Aerospace Engineering} \hfill \textbf{Spring 2022} \\
	Iowa State University
	
	\sectionline
	
	\section*{Relevant Coursework}
	
	\begingroup
	\setlength{\tabcolsep}{8pt}
	\renewcommand{\arraystretch}{1.05}
	\begin{tabular*}{\linewidth}{@{\extracolsep{\fill}} l l l}
		Orbital Mechanics I/II/III & Spacecraft Systems & Space Situational Awareness \\
		Spacecraft Attitude Dynamics and Control & Satellite Communications & Electronic Space Propulsion \\
		Nonlinear Programming & Optimization Theory & \\
	\end{tabular*}
	\endgroup
	
	\sectionline
	
	\section*{Skills}
	\textbf{Programming Languages:} MATLAB, Python, C/C++, Ruby \\
	\textbf{Frameworks / Libraries / Tools:} NumPy, SciPy, Matplotlib, Astropy, Poliastro, bpy, git \\
	\textbf{Applications:} Simulink, Blender
	
	\sectionline
	
	\section*{Work Experience}
	
	\textbf{Student Co-Op, Electronics for Contested Space Group} \hfill \textbf{September 2025 – Present} \\
	\emph{MIT Lincoln Laboratory}
	\begin{itemize}
		\item Implemented an \textbf{\emph{Unscented Kalman Filter (UKF)}} for precise radio frequency measurements and state estimation
		\item Built a probabilistic detection tool to compute observation likelihoods based on resident space object properties and optical sensor performance
	\end{itemize}
	
	\textbf{Guidance, Navigation \& Controls Engineer Intern} \hfill \textbf{May 2025 – August 2025} \\
	\emph{Blue Canyon Technologies}
	\begin{itemize}
		\item Verified functionality and polarity of IMU, Nano Star Tracker, Reaction Wheels, Torque Rods, and Sun Sensors via hardware testing and data checks
		\item Performed regression analysis of two-axis \textbf{\emph{Solar Array Drive Assembly (SADA)}} momentum management and validated command interfaces for precise control and reliability
		\item Automated \textbf{\emph{SADA}} validation by developing \textbf{\emph{Ruby}} test scripts and mapping telemetry channels to \textbf{\emph{COSMOS}}
	\end{itemize}
	
	\textbf{Guidance, Navigation \& Controls Engineer Intern} \hfill \textbf{January 2025 – April 2025} \\
	\emph{Blue Origin}
	\begin{itemize}
		\item Integrated \textbf{\emph{Active Disturbance Rejection Control (ADRC)}} and \textbf{\emph{Sliding Mode Control (SMC)}} to develop a robust algorithm for stabilizing nonlinear MIMO dynamics of the BE-7 engine
		\item Evaluated control performance by injecting disturbances and demonstrated improved accuracy in setpoint tracking
		\item Integrated flight software into \textbf{\emph{Simulink}} using \textbf{\emph{S-functions}} in \textbf{\emph{C}} to enable testing and verification
	\end{itemize}
	
	\textbf{Guidance, Navigation \& Controls Engineer Intern} \hfill \textbf{May 2024 – August 2024} \\
	\emph{Varda Space Industries}
	\begin{itemize}
		\item Conducted trade studies to optimize gravity models for mission requirements and select optimal filter type (\textbf{\emph{EKF}} vs. \textbf{\emph{UKF}})
		\item Built Monte Carlo simulations to quantify reentry uncertainty, generating latitude/longitude covariance ellipsoids and a reentry dispersion cloud for flight safety analysis and capsule recovery planning
		\item Implemented an \textbf{\emph{EKF}} for state estimation, optimizing ground station timing for minimal residuals and precise delta-v planning
		\item Added unit tests and CI/CD pipelines using Bamboo for continuous integration
	\end{itemize}
	
	\textbf{Guidance, Navigation \& Controls Engineer Intern} \hfill \textbf{May 2023 – August 2023} \\
	\emph{Space Dynamics Laboratory}
	\begin{itemize}
		\item Implemented a UKF with range iteration and least-squares orbit determination methods using optical navigation
		\item Performed Monte Carlo analysis on relative orbits to identify challenging scenarios and refine estimation algorithms
		\item Developed unit tests for \textbf{\emph{Lambert Solver}} integrated with \textbf{\emph{Initial Orbit Determination (IOD)}} pipeline
	\end{itemize}
	
	\newpage
	
	\sectionline
	
	\section*{Research Projects}
	
	\textbf{Global Trajectory Optimization Competition (GTOC 6 \& 11) – Modeling \& Simulation} 
	\begin{itemize}
		\item Developed high-fidelity simulations for interplanetary trajectory design and optimization problems under nonlinear dynamics
		    \item Contributed to solution verification and analysis workflows; maintained analysis notebooks and documentation for team reproducibility
	\end{itemize}
	
	\textbf{Delta-V Minimization from Geostationary Orbit to Mars}
	\begin{itemize}
		\item Applied trajectory optimization techniques to minimize delta-v for an Earth-to-Mars transfer, improving fuel efficiency
		\item Generated pork-chop plots using Lambert solutions and validated optimizer results against global minima
		\item Visualized optimized trajectories and planetary motion via Blender’s Python API
	\end{itemize}
	
	\sectionline
	
	\section*{Teaching \& Mentoring Experience}
	
	\textbf{Youth Development Professional} \hfill \textbf{October 2024 -- January 2025} \\
	\emph{Boys \& Girls Club}
	\begin{itemize}
		\item Designed and taught an after-school \textbf{computer literacy} curriculum for grades 3–6 (typing, file systems, internet safety).
		\item Introduced \textbf{beginner coding} using Scratch and Python through project-based lessons.
		\item Differentiated supports to meet diverse learning needs and sustain engagement.
	\end{itemize}
	
	\textbf{Math Teacher (Algebra I, Algebra II, Precalculus, Calculus)} \hfill \textbf{October 2024 -- January 2025} \\
	\emph{North Star School}
	\begin{itemize}
		\item Developed \textbf{unit plans, assignments, and assessments} aligned with course objectives and STEM applications.
		\item Integrated \textbf{Desmos/GeoGebra} to strengthen conceptual understanding.
		\item Provided individualized feedback and academic support to improve performance.
	\end{itemize}
	
	\textbf{Research Assistant, AE 298 National Defense Education Program} \hfill \textbf{January 2023 -- May 2024} \\
	\emph{Aerospace Engineering, University of Illinois Urbana–Champaign}
	\begin{itemize}
		\item Co-taught an introductory \textbf{rocketry} course with hands-on labs and launch activities.
		\item Collected and analyzed assessment data to evaluate learning outcomes and program impact.
		\item Contributed to course redesign and rubric updates to boost engagement and retention.
	\end{itemize}
	
	\textbf{Research Assistant, Grants for Advancement of Teaching in Engineering} \hfill \textbf{November 2023 -- May 2024} \\
	\emph{Mechanical Engineering, University of Illinois Urbana–Champaign}
	\begin{itemize}
		\item Developed a comprehensive \textbf{final project framework} using interdisciplinary pedagogy and clear assessment criteria.
		\item Researched and authored a \textbf{literature review}; contributed to a conference manuscript and instructional materials.
	\end{itemize}
	
\end{document}
