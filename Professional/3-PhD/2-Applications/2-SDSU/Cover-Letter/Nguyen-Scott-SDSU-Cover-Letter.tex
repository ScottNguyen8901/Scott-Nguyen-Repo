\documentclass[11pt]{article}
\usepackage[margin=1.5in]{geometry}
\usepackage{parskip}
\usepackage[hidelinks]{hyperref}
\usepackage[T1]{fontenc}
\usepackage[utf8]{inputenc}
\usepackage{fancyhdr}

% Header setup
\pagestyle{fancy}
\fancyhf{}
\rhead{
	\begin{tabular}{r}
		\textbf{Scott Nguyen}\\
		\href{mailto:scott.nguyen8901@gmail.com}{scott.nguyen8901@gmail.com}\\
		(515) 943--6283\\
		\href{https://www.linkedin.com/in/scottnguyen8901/}{linkedin.com/in/scottnguyen8901}
	\end{tabular}
}
\renewcommand{\headrulewidth}{0pt}

\begin{document}
	
	\vspace*{1.5cm}
	
	October 14, 2025
	
	\textbf{Dr. Pablo Machuca}\\
	Department of Aerospace Engineering\\
	San Diego State University\\
	\href{mailto:pmachuca@sdsu.edu}{pmachuca@sdsu.edu}
	
	Dear Dr. Machuca,
	
	I am writing to express my interest in the Ph.D. position in autonomous Guidance, Navigation, and Control for cislunar space within the SDSU--UCSD Joint Doctoral Program. I am currently completing my M.S. in Electrical/Space Systems Engineering at the University of New Mexico and hold an M.S. in Aerospace Engineering from the University of Illinois Urbana--Champaign. I’m eager to expand my work in state estimation, control, and trajectory design into the challenging cislunar domain.
	
	Over the past several years, I’ve gained experience across startups, large aerospace companies, university-affiliated research centers, and a federally funded research center. My work has included orbit determination with sparse optical measurements, Kalman-filter--based state estimation, adaptive model-free disturbance rejection for lunar lander engine control, stochastic methods for resident-space-object detection, and hardware-in-the-loop testing for commercial, civil, and government missions. I also contributed to modeling and simulation efforts for the Global Trajectory Optimization Competitions (GTOC 6 and 11), strengthening my skills in high-fidelity dynamics, Monte Carlo analysis, and uncertainty quantification using MATLAB, Python, and C/C++.
	
	I’m particularly drawn to this opportunity because it unites my interests in lunar exploration, sensor modeling, and autonomous navigation. I hope to build upon my background in estimation and simulation to develop new approaches for multi-sensor integration, feature detection, and CR3BP-based trajectory design. I’m excited about the possibility of contributing to your group’s efforts in advancing autonomous GNC for cislunar missions and would welcome the opportunity to discuss how my experience and goals align with your research vision.
	
	Sincerely,\\[1.5em]
	\textbf{Scott Nguyen}
	
\end{document}
