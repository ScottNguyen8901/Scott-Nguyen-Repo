%&latex
\documentclass[12pt,legalpaper]{article}
\usepackage[plain]{fullpage}

\usepackage{amsmath}
\usepackage{amssymb}
\usepackage{amsfonts}
\usepackage[lined,algonl,boxed]{algorithm2e}

\begin{document}

\vspace{2in}
\begin{center}
  {\Huge\bf ECE 506 Optimization Theory}
\end{center}
\vspace{0.5in}
%  
{\bf\large
\begin{flushleft}
 \hspace{0.3in} Name: \qquad \underline{\hspace{3in}}\\[0.5in]
 \hspace{1.5in}Problem 1: \underline{\hspace{0.3in}} /15 \\[0.1in]
 \hspace{1.5in}Problem 2: \underline{\hspace{0.3in}} /20 \\[0.1in]
 \hspace{1.5in}Problem 3: \underline{\hspace{0.3in}} /10 \\[0.1in]
 \hspace{1.5in}Problem 4: \underline{\hspace{0.3in}} /30 \\[0.1in]
 \hspace{1.5in}Problem 5: \underline{\hspace{0.3in}} /25 \\[0.5in]
 \hspace{1.5in}Total: \phantom{\hspace{0.4 true in}}
                     \underline{\hspace{0.3in}} /100
\end{flushleft}}
\vspace{2in}
{\bf\Huge
\begin{center}
  Good Luck!
\end{center}}
\newpage
%
%
\noindent{\bf Problem 1 ({\bf 15 points})}\\
\noindent 1(a)({\bf 5 points}) Consider the cubic function given by
        $$f(x_1, x_2) = a (x_1-1)^3 + b (x_2-2)^3. $$
Give expressions for $\nabla f$ and $\nabla^2 f$.\\
\vspace{3.5 true in}

\noindent 1(b)({\bf 5 points}) When is $f$ stationary?\\
\vspace{1.0 true in}

\noindent 1(c)({\bf 5 points} At the stationary point, is it possible to
 gave a strict minimum? If yes, when?
\newpage
%
%
\noindent{\bf Problem 2 (20 points)}\\
We begin by restating theorem 4.1 from your text.
The vector $p^*$ is a global solution of the trust-region problem:
$$ \min_{p\in R^n} m(p) = f+g^T p + \frac{1}{2} p^T B p, 
   \quad s.t. \quad ||p||\leq \Delta,$$
if and only if $p^*$ is feasible and there is a scalar $\lambda\geq 0$
such that the following conditions are satisfied:
\begin{enumerate}
\item $(B+\lambda I)p^* = -g$ 
\item $\lambda (\Delta - ||p^*||)=0$
\item $(B+\lambda I)$ \quad is positive semidefinite.
\end{enumerate}

\noindent 2(a)({\bf 5 points}) First, let us consider the unconstrained problem.
Give an optimal expression for $p^*$ by solving $\nabla m(p^*)=0$.
\vspace{1.0 true in}

\noindent 2(b)({\bf 5 points}) Suppose that $||p^*||<\Delta$ lies inside the feasible region. Show that this can only happen if $B$ is positive
semidefinite. Furthermore, how does the optimal $p^*$ compare with the 
one found in 2(a)?
\vspace{1.0 true in}

\noindent 2(c)({\bf 5 points}) Suppose that the solution is on the boundary:
$||p^*||=\Delta$ and $\lambda>0$. Derive an expression for $p^*$ in 
terms of $\nabla m(p)$.
\vspace{1.0 true in}


\noindent 2(d)({\bf 5 points}) Based on your answer in 2(c), sketch $p^*$, the trust region, and the contours of $m(p)$.
\newpage


\vspace{1.0 true in}





%

\noindent{\bf Problem 3 (10 points)}.\\
Suppose that $f$ is twice differentiable and that the Hesian $\nabla^2 f(x)$ is Lipschitz continuous
   in a neighborhood of the solution $x^*$, at which the second order 
   sufficient conditions for a strict
   minimum are satisfied.
Consider the Newton algorithm:
\begin{align*}
  x_{k+1} &= x_k + p_k \\
  p_k^N   &= -\nabla^2 f_k^{-1} \nabla f_k.
\end{align*}      
Assuming that the starting point is sufficiently close to $x^$, 
  we can show that $\{x_k\}$ converges to $x^*$ quadratically. You
are not asked to prove this. You need to show that
 $\{||\nabla f_k||\}$ converges quadratically to zero.\\~\\   
{\it Hint}: Start from:
       $$||\nabla f(x_{k+1})||=||\nabla f(x_{k+1}) - 0||,$$
replace $0$ by a suitable expression and then use the mean-value theorem
for the gradient:
 $$\nabla f(x+p) = \nabla f(x)
            + \int\limits_0^1 \nabla^2 f(x+tp)p \, dt, \quad t\in(0,1).$$
\newpage

\noindent{\bf Problem 4 (30 points)}\\
\noindent{\bf 4(a)(4 points)} Consider $a^T b$ where $a,\, b$ are 
  $n$-dimensional vectors. Give:
  \begin{itemize}
\item $M:$ the number of multiplications. 
\item $A:$ the number of additions.
\item $S:$ the storage requirements for $B$-bytes per vector element
           (only for $a,b$).
\item $m:$ the number of memory accesses to the elements of $a,b$.      
\end{itemize}
\vspace{2.0 true in}

\noindent{\bf 4(b)} Consider the following algorithm:
\incmargin{1em}
\restylealgo{boxed}\linesnumbered
\begin{algorithm}
  $q \leftarrow \nabla f_k$\\
  \For{i=k-1, \, k-2, \dots, \, k-m}{
   $\alpha_i \leftarrow \rho_i s_i^T q;$\\
   $q \leftarrow q - \alpha_i y_i$
  }
  $r \leftarrow H_k^0 q;$\\
  \For{i=k-m,\, k-m+1, \, \dots, \, k-1}{
    $\beta\leftarrow \rho_i y_i^T r;$\\
    $r\leftarrow r+s_i(\alpha_i-\beta);$
  }  
\caption{L-BFGS two-loop recursion for computing $r=H_k\nabla f_k$.}
\end{algorithm}
\decmargin{1em}\\
Give a list of all of the input variables (2 points):\\~\\

\noindent What are the storage requirements for representing $H_k$ (4 points)?:

\newpage
\noindent Without considering the cost of computing $r\leftarrow H_k^0 q$, based on your answer in (a), give the following (10 points):
\begin{itemize}
 \item $T_M:$ the total number of multiplications
 \item $T_A:$ the total number of additions
 \item $Tsub:$ the total number of subtractions
 \item $Tmem:$ the number of memory accesses
\end{itemize} 
\vspace{6.0 true in}

\noindent{\bf 4(c)(5 points)} Assume that we initialize using 
$H_k^0=\gamma_k I$, where $\gamma_k=s_{k-1}^T y_{k-1}/(y_{k-1}^T y_{k-1})$.
Based on (a), give the number of multiplications and additions needed to compute $r\leftarrow H_k^0 q$.\\
\newpage

\incmargin{1em}
\restylealgo{boxed}\linesnumbered
\begin{algorithm}
Choose starting point $x_0$, integer $m>0;$\\
$k\leftarrow 0;$\\
\Repeat(){convergence}{
Choose $H_k^0$\\
Compute $p_k=-H_k\nabla f_k$ using L-BFGS two-loop recursion algorithm.\\
Compute $x_{k+1}\leftarrow x_k + \alpha_k p_k,$ where $\alpha_k$ is \\
$\qquad$  chosen so as to satisfy the Wolfe conditions;\\
\If{$k>m$}{Discard the vector pair $\{s_{k-m},\, y_{k-m}\}$ from storage;}
Compute and save 
$s_k\leftarrow x_{k+1}-x_k,\, y_k\leftarrow\nabla f_{k+1}-\nabla f_k, \, \rho_k\leftarrow 1/(y_k^T s_k)$.\\
$k\leftarrow k+1;$
}
\caption{L-BFGS.}
\end{algorithm}
\decmargin{1em}
\noindent{\bf 4(d)} In the L-BFGS algorithm given above, assume
that it takes $P$ function evaluations for $f_k$ and $Q$ function evaluations
for $\nabla f_k$ to find $\alpha_k$ that satisfies the Wolfe conditions.
Please
note that $\alpha_k=1$ is expected to work. For each iteration, give (3 points):
\begin{itemize}
 \item $A_f:$ the average number of  function evaluations
 \item $A_{Df}:$ the average number of $\nabla f_k$ evaluations
\end{itemize}\\
\vspace{2.0 true in}

\noindent{4(e)(2 points)} Assume that each gradient component is estimated
using
 $$\frac{f_i}{\partial x_i}(x) \approx \frac{f(x+\epsilon e_i)-f(x)}{\epsilon},$$
where $\epsilon$ is chosen as discussed on page 196 of your book (not discussed here). In this case, give $A_f$ accounting for the average number of function
evaluations in 4(d).

\newpage
\noindent{\bf Problem 5 (25 points) Stochastic Optimization}\\
\noindent {\bf 5(a)(4 points)} We want to minimize $f(x)$
  where $x$ is a vector of {\bf non-negative integers}:
      $$(x_1, x_2, \dots, x_p).$$
  Let each integer $x_i$ be represented with $b$-bits.
  For $p$-dimensional vectors, how many possible integer vectors do we
  have? Give an example for $p=64$, $b=8$.\\
  \vspace{3.0 true in}


      
\noindent {\bf 5(b)(4 points)} Define a neighborhood system that changes
  a {\bf single} variable at a time by adding or subtracting $1$. Thus, 
  the following three vectors will be neighbours of $(x_1, x_2, \dots, x_p)$:
      $$(x_1+1, x_2, \dots, x_p), (x_1-1, x_2, \dots, x_p).$$
  Here, for simplicity, we assume wrap-around at the edges (eg: in $8$-bits, $255+1$ gets mapped to $0$). Give $|N(x)|$, the number of neighboring
  vectors for $x$. Will this neighborhood system reach all possible vectors?
  Explain briefly. Note that it is clear how to expand this approach to arbitrary regions by simply mapping the results from the current approach to any other range.\\
  \vspace{4.0 true in}
  \newpage
  
%  
\noindent{\bf 5(c)(4 points)} Recall the Markov-chain transition probability expression: 
  \begin{equation*}
      p = \min \left\{1, \,
                 \frac{\exp{(\lambda_n f(y))}/|N(y)|}{
                          \exp{(\lambda_n f(x))}/|N(x)|}
              \right\}.
  \end{equation*}
Provide a simplified expression for $p$ for this case.\\
\vspace{3.0 true in}

\noindent{\bf 5(d)(4 points)} Provide a simple {\bf for}-loop
that can be used to maximize $f(.)$ {\it without storing} all
of the generated values. Your code should terminate after 
{\it MaxIterations} and also specify $\lambda_n$.
\vspace{3.0 true in}   
\newpage

\noindent {\bf 5(e)(9 points)} Suppose that you are asked to provide a more
 global approach that will be used to generate $Q$ starting integer vectors,
 based on a 1-D uniform distribution on each component. {\it Hint:} Generate
 each $x=(x_1, x_2, \dots, x_p)$ by generating each component in the sequence: $x_1, x_2, \dots, x_p$.
 
 Give the pseudo-code for generating the integer vectors {\bf and} how to
 combine the results from each of the $Q$-runs to get the global maximum.
\end{document}
