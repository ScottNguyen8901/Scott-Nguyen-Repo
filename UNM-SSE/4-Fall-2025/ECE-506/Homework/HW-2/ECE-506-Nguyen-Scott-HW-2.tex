\documentclass[11pt]{article}

% Layout & header
\usepackage[margin=1in,headheight=16pt]{geometry}
\usepackage{fancyhdr}
\usepackage{parskip} % no paragraph indents

% Math, figures, lists
\usepackage{amsmath,amssymb,mathtools}
\usepackage{graphicx}
\usepackage{float}
\usepackage{enumitem}

% Code (Python)
\usepackage[dvipsnames]{xcolor}
\usepackage{listings}
\lstdefinestyle{python}{
	language=Python,
	basicstyle=\ttfamily\small,
	keywordstyle=\color{NavyBlue}\bfseries,
	commentstyle=\color{ForestGreen}\itshape,
	stringstyle=\color{BrickRed},
	numbers=left, numberstyle=\tiny, numbersep=8pt,
	frame=single,
	breaklines=true,
	columns=fullflexible,
	showstringspaces=false
}
\lstset{style=python}

% Hyperlinks (load last)
\usepackage{hyperref}

% Header content
\pagestyle{fancy}
\fancyhf{} % clear defaults
\fancyhead[L]{September 22, 2025}
\fancyhead[C]{ECE 506: Homework \#2}
\fancyhead[R]{Scott Nguyen}
\renewcommand{\headrulewidth}{0.4pt}

\begin{document}
	
\textbf{Problem \#1. Quadratic models in 1-D Optimization.}
		
\begin{enumerate}[label=1(\alph*)]
	\item Locally, optimization methods consider a local linear or quadratic model. Consider the quadratic model:
	\[
	f(x) = a \cdot x^{2} + b \cdot x + c
	\]
	Compute a general expression for the extreme point.
	
	\textbf{Solution:} 
	Extreme points occur where the derivative equals zero. Differentiate $f(x) = a x^{2} + b x + c$:
	\[
	f'(x) = 2 a x + b.
	\]
	Set $f'(x) = 0$ and solve:
	\[
	x^* = -\frac{b}{2 a}.
	\]
	
	\item When is $f$ convex?

	\textbf{Solution:} 
	A function is convex when its second derivative is nonnegative. Differentiate twice:
	\[
	f''(x) = 2 a.
	\]
	Set $f''(x) \ge 0$ and solve:
	\[
	a \ge 0.
	\]
	
	\item When is $f$ concave?
	
	\textbf{Solution:} 
	By flipping the inequality from the convex case, $f$ is concave when $a \le 0$.
	
	\item When is the extreme point an actual minimum?
	\textbf{Solution:}  
	The extreme point is a minimum when $f'(x) = 0$ and $f''(x) > 0$, i.e., when $a > 0$.
	
	\item When is the extreme point a maximum?
	
	\textbf{Solution:}  
	The extreme point is a maximum when $f'(x) = 0$ and $f''(x) < 0$, i.e., when $a < 0$.
	
	\item Consider the constraint optimization problem:
	\[
	\min_{x} f(x) \quad \text{subject to: } d \le x \le e.
	\]
	where $-\infty < d < e < \infty$. Based on the KKT conditions, we know that the solution is either at $x = d$ or
	$x = e$ or at the extremum point. Suppose that $a < 0$. Show that the solution is either at $x = d$ or $x = e$. In
	this negative curvature example, the solution is always at the boundary.

	\textbf{Solution:}  
	For $a<0$, $f$ is concave, so any interior critical point is a maximum.  
	From that peak the function decreases toward both ends, so the lowest value must occur at one of the endpoints:
	\[
	x^* \in \{d, e\}.
	\]

	\item For $a > 0$, show that all three cases are possible in (f).
	
	\textbf{Solution:}  
	For $a>0$, $f$ is convex and has a minimum at $x^*=-\dfrac{b}{2a}$.  
	Compare $x^*$ with the interval $[d,e]$:
	\[
	\begin{aligned}
		&\text{If } d \le x^* \le e, && \text{the minimum is at } x^*.\\
		&\text{If } x^* < d,        && \text{the minimum is at } x=d.\\
		&\text{If } x^* > e,        && \text{the minimum is at } x=e.
	\end{aligned}
	\]
	Since all three positional relationships can occur for suitable coefficients, every case is possible.
	
	\end{enumerate}
	
	\noindent\textbf{Notes:} A function $f$ is concave if $-f$ is convex. Use the fact that a function is convex if
	\[
	\frac{\partial^{2} f(x)}{\partial x^{2}} > 0
	\]
	everywhere. Furthermore, note the property of convex functions that
	\[
	f(t x_{1} + (1 - t) x_{2}) \le t f(x_{1}) + (1 - t) f(x_{2}), \quad 0 \le t \le 1.
	\]
	This property implies that convex functions stay below a line that connects the end-points at $x_{1}$ and $x_{2}$.

\end{document}
