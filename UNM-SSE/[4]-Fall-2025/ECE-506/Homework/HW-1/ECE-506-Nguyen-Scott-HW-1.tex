\documentclass[11pt]{article}
\usepackage[margin=1in]{geometry}
\usepackage{amsmath, amssymb, mathtools}
\usepackage{enumitem}
\usepackage{hyperref}

\title{ECE 506: Homework \#1: Basic Optimization}
\date{}

\begin{document}
	\maketitle
	
	\noindent
	To get help with the homework, please join the Saturday morning discussion sessions starting at 9am at
	\url{https://unm.zoom.us/j/99977790315}.
	
	\section*{Problem \#1. An Introduction to Linear Programming}
	
	This problem is focused on manipulating the basic Linear Programming equation:
	\begin{equation}
		\min_{x}\; c^{\top}x
		\quad \text{subject to } Ax = b \text{ and } x \ge 0.
		\label{eq:lp}
	\end{equation}
	(Here, $x \ge 0$ is understood componentwise.)
	
	\begin{enumerate}[label=\textbf{1(\alph*)}]
		\item \textbf{We begin with the simplest possible example.} Consider the 1D problem:
		\begin{equation}
			\min_{x}\; c \cdot x
			\quad \text{subject to } a x = b \text{ and } x \ge 0.
			\label{eq:1d}
		\end{equation}
		From this case, answer the following:
		\begin{enumerate}[label=\roman*)]
			\item \textbf{Example with no solution.}
			With the constraints \(ax=b\) and \(x\ge 0\), if \(a\neq 0\) then the only candidate is
			\(x^\star=\tfrac{b}{a}\).
			If \(b/a<0\), the nonnegativity constraint is violated, so the problem is infeasible; e.g.,
			\(a=1,\ b=-1 \Rightarrow x^\star=-1\) (infeasible).
			(Also infeasible when \(a=0,\ b\neq 0\) since \(0=b\) cannot hold.)
			
			\item \textbf{Example with a simple solution.}
			Take \(a=2,\ b=0\). Then \(x^\star=\tfrac{b}{a}=0\), which satisfies \(x\ge 0\), and the objective value is \(c\,x^\star=0\).
			
			\item \textbf{Did you minimize anything? Explain.}
			No. When \(a\neq 0\), the equality constraint pins down a single feasible point \(x^\star\);
			if it is feasible, it is automatically optimal—there is no tradeoff to optimize over.
		\end{enumerate}

		\item \textbf{Invertible case.} If $A$ is invertible, the constraint $Ax=b$ has the unique solution $x^\star=A^{-1}b$. If $x^\star\ge 0$ (componentwise), it is the only feasible—and thus optimal—point with value $c^\top x^\star$; otherwise the problem is infeasible. No minimization needed.

		\item \textbf{Underdetermined case.} The only case that is interesting is when we have many solutions to $Ax = b$.
		We then get to pick the one that minimizes $c^{\top}x$. This can only happen when
		the number of equations is smaller than the number of unknowns. Here is an example:
		\[
		\begin{bmatrix} 1 & 2 \end{bmatrix}
		\begin{bmatrix} x_1 \\ x_2 \end{bmatrix} = 2.
		\]
		Note that we have one equation in two unknowns. We have more unknowns
		than we have equations! It may be possible to set up a proper optimization
		problem.
		
		\medskip
		To have a proper solution, we must also satisfy $x_1, x_2 \ge 0$. These are called
		\emph{feasible solutions}. They satisfy the constraints, and the optimal solution needs
		to satisfy them.
		
		\medskip
		\textbf{Task:} Plot all possible solutions of $Ax = b$ satisfying $x_1, x_2 \ge 0$ for this case.
		
		\item \textbf{Optimization over the feasible set.} For the case when $Ax = b$ described in 1(c), solve the proper optimization
		problem. For this case, solve:
		\begin{equation}
			\min_{x}\; [\,1\;\;1\,]\,x \quad \text{subject to } Ax = b \text{ and } x \ge 0.
			\label{eq:obj11}
		\end{equation}
		Is the solution at the endpoints? Explain.
	\end{enumerate}
	
\end{document}
